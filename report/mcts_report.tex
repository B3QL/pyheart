\documentclass[10pt,a4paper]{article}
\usepackage[utf8]{inputenc}
\usepackage[polish]{babel}
\usepackage[T1]{fontenc}
\usepackage[left=2cm,right=2cm,top=2cm,bottom=2cm]{geometry}
\author{Bartłomiej Kurzeja}
\title{Wybrane zagadnienia sztucznej inteligencji \\ Ćwiczenie 1: MCTS, Gry niedeterministyczne}
\begin{document}
\maketitle
\section{Wstęp}
\subsection{Implementacja MCTS}
Każdy węzeł drzewa reprezentują pojedyńczą akcję gracza taką jak:
\begin{itemize}
	\item zagranie karty,
	\item zagranie karty na miniona (swojego lub przeciwnika),
	\item atak miniona przeciwnika lub samego przeciwnika,
	\item zakończenie tury. 
\end{itemize}

Dodatkowo przyjętym kryterium wyboru najlepszego dziecka do eksploracji jest Upper Confidence Bounds for Trees (UCT) ze stała $C_p = \frac{1}{\sqrt{2}}$. Natomiast najlepszy ruch wyznaczany jest na podstawie najwyższej wygranej dziecka, tak zwany Max Child.

Implementacja obejmuję również węzły nie deterministyczne w tym przypadku prawdopodobieństwo zagrania karty przez gracza jest równy stosunkowi liczby kart w ręce do liczby kart jeszcze nie zagranych.

W celu polepszenia wyników drzewo MCTS nie jest budowane każdorazowo od nowa, lecz wybierane jest poddrzewo na podstawie przebiegu gry do rozbudowy.

\subsection{Implementacja wybranych kart}
\paragraph{Dire Mole} 1 koszt / 1 atak / 3 zdrowie
\paragraph{River Crocolisk} 2 koszt / 2 atak / 3 zdrowie
\paragraph{Magma Rager} 3 koszt / 5 atak / 1 zdrowie
\paragraph{Chillwind Yeti} 4 koszt / 4 atak / 5 zdrowie
\paragraph{Stormpike Commando} 5 koszt / 4 atak / 2 zdrowie - Zadaje 2 obrażenia przy zagraniu
\paragraph{Boulderfist Ogre} 6 koszt / 6 atak / 7 zdrowie
\paragraph{Flamestrike} 7 koszt - Zaklęcie zdające 4 obrażenia wszystkim minionom przeciwnika
\paragraph{Dinosize} 8 koszt - Zaklęcie ustawiające 10 punktów zdrowia i ataku wybranemu minionowi
\paragraph{King Krush} 9 koszt / 8 atak / 8 zdrowie - Zdolność do natychmiastowego ataku
\paragraph{Goldthorn} 10 koszt - Zaklęcie dodające 6 punktów zdrowia minonowi

\subsection{Funkcja celu graczy zachłannych}
\paragraph{Gracz agresywny} wybiera kolejno: atak na gracza, atak na dowolnego minona, losowa akcja.
\paragraph{Gracz kontrolujący} wybiera kolejno: atak na dowolnego miniona, losowa akcja.
\section{Badania}
\subsection*{Założenia}
\begin{itemize}
	\item Łączna liczba gier przypadająca na typ gracza wynosi 200.
	\item Liczba playoutów jest stała i wynosi 30 (czas wykonania około 10 sekund).
	\item Głębokość drzewa dla korzenia wynosi 1.
	\item Średni procent eksploracji drzewa to średnia wartość stosunku liczby liści do liczby wszystkich dzieci we wszystkich węzłach.
\end{itemize}
\subsection*{Wyniki}
\begin{table}[h]
\centering
\caption{Czas budowania drzewa w zależności od liczby playoutów.}
\begin{tabular}{|l|l|l|l|l|l|}
\hline
Liczba playoutów & 10     & 30    & 50     & 100    & 1000       \\ \hline
Czas wykonania   & 4.21 s & 9.6 s & 13.1 s & 33.3 s & 5 min 32 s \\ \hline
\end{tabular}
\end{table}

\begin{table}[h]
\centering
\caption{Efektywność gracza MCTS}
\begin{tabular}{|l|c|c|c|}
\hline
Typ przeciwnika             & Losowy & Agresywny & Kontrolujący \\ \hline
Procent wygranych           & 99 \%  & 71 \%     & 97 \%        \\ \hline
Minimalna głębokość liścia  & 1      & 1         & 1            \\ \hline
Średnia głębokość liścia    & 3.70   & 3.63      & 3.85         \\ \hline
Maksymalna głębokość liścia & 8      & 7         & 7            \\ \hline
Mediana głębokości liścia   & 4      & 4         & 4            \\ \hline
Średni procent eksploracji  & 11 \%  & 11 \%     & 12 \%        \\ \hline
\end{tabular}
\end{table}
\end{document}